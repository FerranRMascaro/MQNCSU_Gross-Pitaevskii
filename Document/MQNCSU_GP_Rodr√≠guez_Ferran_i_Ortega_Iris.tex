\documentclass[a4paper]{article}

\usepackage[utf8]{inputenc}
\usepackage{lmodern}
\usepackage[T1]{fontenc}
\usepackage[catalan]{babel}
\usepackage{mathtools}
\usepackage{upgreek}
\usepackage{float}
\usepackage{csvsimple}
\providecommand{\abs}[1]{\lvert#1\rvert}

% \usepackage[backend=biber,style=phys]{biblatex}
% \addbibresource{.bib}

% \usepackage{multirow}
\usepackage{multicol}
\usepackage{multirow}
\usepackage[table,xcdraw]{xcolor}
\usepackage{hhline}
\usepackage{rotating}
\usepackage{tikz}
\usepackage{graphicx}
\usepackage{wrapfig}

% \usepackage{graphicx}
% \usepackage{caption}
% \usepackage{subcaption}

% \usepackage[a4paper]{geometry}
% \geometry{top=3cm, bottom=3.3cm, left=3cm, right=3cm}
\usepackage{titling}
% \renewcommand{\arraystretch}{1.8} 
% \newcolumntype{M}{X<{\vspace{4pt}}} 
\usepackage{gensymb} %to use degrees
\usepackage{fancyhdr}
\usepackage[hidelinks]{hyperref}
\pagestyle{fancy}

\usepackage{braket}


\lhead{\bf MQDNCISU}
\rhead{Iris Ortega i Ferran Rodríguez}
\title{Entrega: Weakly interacting and confined bosons at low density}
\author{Iris Ortega i Ferran Rodríguez}


\begin{document}
\maketitle{}


\subsection*{Enunciat}

\textbf{The theoretical description of the recent experiments on Bose Einstein condensation (BEC) is based on the Gross-Pitaevskii (GP) equation. The atoms are confined by a magnetic field, which effects are very well described by an harmonic oscillator potential. Here we assume a condensate of $\boldsymbol{^{87}Rb}$. We made the assumption that all the atoms are in the condensate, $\boldsymbol{\Psi(\vec{r})}$ normalised such that:}
\begin{equation}
    \boldsymbol{\int d\vec{r}\ |\Psi(\vec{r})|^2=1 \ .}
\end{equation}
\textbf{In harmonic oscillator units, the GP equation reads:}
\begin{equation}
    \boldsymbol{\left[-\dfrac{1}{2}\nabla^2 +\dfrac{1}{2}r_1^2+4\pi \bar a_s N |\bar\Psi(\vec{r}_1)|^2\right]\bar\Psi(\vec{r}_1)=\mu \bar\Psi(\vec{r}_1)\ ,}
\end{equation}
\textbf{where $\boldsymbol{\mu}$ is the chemical potential, $\mathbf{N}$ the number of particles and $\mathbf{a_s}$ the $\mathbf{s}$-wave scattering length in harmonic oscillator units. In this case, $\boldsymbol{\bar\Psi(\vec{r}_1)}$ is normalised to $\mathbf{1}$.}\\

Per fer l'exercici s'ha escrit un nou programa amb \textit{Python} i les llibreries \texttt{math}, \texttt{argparse}, \texttt{numpy} i \texttt{pandas}. L'hem anomenat \texttt{Iris\_Ortega-Ferran\_Rodriguez.py}. Tots els resultats exposats en aquest treball obtinguts amb aquest programa, han estat comprovats amb els obtinguts amb el programa de \textit{Fortran} proporcionat. Per saber com funciona el nou programa es pot introduir al terminal:
\begin{quote}
    \texttt{python3 Iris\_Ortega-Ferran\_Rodriguez.py -h}.
\end{quote}
Per usar-lo només requereix un argument, el \textit{path} on es troba el fitxer \textit{input}, un fitxer \textit{.txt} amb tantes files com diferents valors de les variables es vulguin introduir i amb columnes corresponents a: la longitud de \textit{scattering} en unitats d'osci"lador (\texttt{a0}), el nombre de passes (\texttt{N\_steps}), la mida del pas (\texttt{step}), el nombre de partícules (\texttt{N}), el pas en el 'temps' (\texttt{time}), el valor d'alfa (paràmetre d'inici per la funció de l'osci"lador harmònic, \texttt{alpha}) i el nombre d'iteracions (\texttt{iter}).


\subsubsection*{a0) Using the GP program check that if you put the interaction equal to zero, then you recover the expected results
for the harmonic oscillator. Why is the energy per particle equal to the chemical potential?}

En el nostre programa, per tenir la solució per l'osci"lador harmònic només cal afegir un darrer argument en la terminal: \texttt{-ao}. Això fa que la variable \texttt{cequ} del programa passi a valdre $0$.

En córrer el programa per un \textit{input} tal com el de la Taula \ref{tab:input_a0}, obtenim $E=1.4999665803921967$ per l'energia i $\mu=1.4999665800446846$ pel potencial químic, ambdós en unitats de l'osci"lador harmònic. La diferència entre ambdós valors és de $10$ ordres de magnitud inferior a aquests: $E-\mu=3\cdot10^{-10}$. Per tant, tenim $E\approx \mu\approx 1.5$. 

\begin{table}[h!]
    \centering
    \begin{tabular}{|c|c|c|c|c|c|c|}
        \hline
        \rowcolor[HTML]{EFEFEF}
        \textbf{\texttt{a0}} & \textbf{\texttt{N\_steps}} & \textbf{\texttt{step}} &\textbf{\texttt{N}} & \textbf{\texttt{time}} & \textbf{\texttt{alpha}} & \textbf{\texttt{iter}} \\ \hline\hline
        $0.00433$ & $700$ & $0.015$ & $1000000$ & $0.00005$ & $0.3$ & $70000$ \\ \hline
    \end{tabular}
\caption{\textit{Input} del programa per l'apartat a0).}
\label{tab:input_a0}
\end{table}

Les dues magnituds valen aproximadament el mateix, ja que en treure el terme corresponent a la interacció, l'equació de Gross-Pitaevski adimensional es redueix a una equació de Schrödinger adimensional per un potencial donat. Els autovalors resultat de l'equació de Schrödinger són les possibles energies del sistema, mentre que els de l'equació de Gross-Pitaevskii són els del potencial químic. Com que la part de l'esquerra d'ambdues equacions s'iguala, el resultat que obtenim per la dreta, els autovalors, és el mateix.

\subsubsection*{a) Take $\mathbf{\bar a_s = 0.0043}$, appropriate for $\mathbf{^{87}Rb}$ and, using the program, solve the Gross-Pitaevskii equation. Study the dependence on the number of particles ($\mathbf{N}=100, 1000, 10000, 100000, 1000000$) of the chemical potential, and the following energies per particle: total, kinetic, harmonic oscillator and interaction energy. Construct a table with the results and comment their behaviour.}

Per aquest apartat hem usat com a \textit{input} el proporcionat en el fitxer \textit{grosspita.grau.input.txt}, que es troba recollit en la Taula \ref{tab:input_a}.

\begin{table}[h!]
    \centering
    \begin{tabular}{|c|c|c|c|c|c|c|}
        \hline
        \rowcolor[HTML]{EFEFEF}
        \textbf{\textit{a0}} & \textbf{\textit{N\_steps}} & \textbf{\textit{step}} &\textbf{\textit{N}} & \textbf{\textit{time}} & \textbf{\textit{alpha}} & \textbf{\textit{iter}} \\ \hline\hline
        0.00433 & 700 & 0.015 & 1000000 & 0.00005 & 0.3 & 70000\\ \hline
        0.00433 & 600 & 0.015 & 100000 & 0.0001 & 0.4 & 50000 \\ \hline
        0.00433 & 400 & 0.015 & 10000 & 0.0001 & 0.8 & 40000 \\ \hline
        0.00433 & 400 & 0.020 & 1000 & 0.0001 & 0.5 & 50000 \\ \hline
        0.00433 & 300 & 0.020 & 100 & 0.0001 & 0.5 & 50000 \\ \hline
    \end{tabular}
\caption{\textit{Input} del programa per l'apartat a).}
\label{tab:input_a}
\end{table}

Els resultats es troben a la Taula \ref{tab:res_a}.

\begin{table}[]
    \centering
    \begin{tabular}{|c|c|c|c|c|c|c|}
    \hline
        \rowcolor[HTML]{EFEFEF}
        $\mathbf{N}$ & $\boldsymbol{\mu}$ & $\boldsymbol{\mu_{\textbf{T}}}$ & $\boldsymbol{E_{\textbf{T}}}$ & $\boldsymbol{E_{\textbf{cin.}}}$ & $\boldsymbol{E_{\textbf{o.h.}}}$ & $\boldsymbol{\mu_{\textbf{inter.}}}$ \\ \hline\hline
        \textbf{1000000} & 42.119 & 30.059 & 30.120 & 0.061 & 18.060 & 11.999 \\ \hline
        \textbf{100000} & 16.847 & 11.980 & 12.104 & 0.124 & 7.238 & 4.743 \\ \hline
        \textbf{10000} & 6.866 & 4.801 & 5.042 & 0.240 & 2.977 & 1.824 \\ \hline
        \textbf{1000} & 3.045 & 1.987 & 2.425 & 0.438 & 1.367 & 0.620 \\ \hline
        \textbf{100} & 1.788 & 0.995 & 1.652 & 0.656 & 0.860 & 0.136 \\ \hline
    \end{tabular}
\caption{Resultats d'introduir al programa l'\textit{input} de la Taula \ref{tab:input_a}. Per cada nombre de partícules al sistema tenim el potencial químic mitjà corresponent, l'energia total, la cinètica, la corresponent a l'osci"lador harmònic i la corresponent al terme d'interacció. S'observa que el potencial total equival a la suma del corresponent a l'osci"lador harmònic més el del terme d'interacció mentre que l'0energia total és el potencial total més l'energia corresponent al terme cinètic.}
\label{tab:res_a}
\end{table}

\subsubsection*{b) Do the same using the Thomas-Fermi approximation. Notice that the kinetic energy in this approach is taken to zero.}


\subsubsection*{c) Make a plot of the density profile $\boldsymbol{\rho (r_1)}$ normalised such that \begin{equation}
 \boldsymbol{\int dr_1 r_1^2 \rho(r_1)=1}
\end{equation} for $\mathbf{N=1000}$ and $\mathbf{N=100000}$ and compare the GP and the TF results.}


\subsubsection*{d) Check numerically that the solutions of the GP equation fulfill the virial theorem for different values of $\mathbf{N}$.}

Per trobar l'equació corresponent al teorema del Virial ens construïm la funció $\Psi_{\lambda}(\{\vec r'\})=\lambda^{3N/2}\Psi_{\text{G.S}}(\{\lambda\vec r\})$, on G.S. vol dir \textit{ground state}. Aleshores, calculem l'autovalor de l'equació de GP:
\begin{equation}
    \langle E(\lambda)\rangle=\int d^{3N}\vec r \ \Psi_{\lambda}^{*}(\{\vec r'\})\mathcal{H} \Psi_{\lambda}(\{\vec r'\}) \ .
\end{equation}

Fent el canvi de variable $\vec r'= \lambda\vec r$, obtenim per partícula: 
\begin{equation}
    \nabla^2_{\vec r}=\lambda^2\nabla^2_{\vec r'}, \quad \vec r^2=\dfrac{{\vec r}^{\ '2}}{\lambda^2}, \quad |\Psi_{\lambda}(\{\vec r'\})|^2=\lambda^3\Psi_{G.S.}|(\{\vec r'\})|^2 \ .
\end{equation}

Substituint aquests valors obtenim:
\begin{equation}
    e= \lambda^2 e_{\text{cin.}}+\dfrac{1}{\lambda^2}e_{\text{o.h.}}+\lambda^3 e_{\text{int.}}\ .
\end{equation}

I imposant condició d'extrem obtenim l'equació del Virial:
\begin{equation}
    \left.\dfrac{de(\lambda)}{d\lambda}\right\vert_{\lambda=1}=2e_{\text{cin.}}-2e_{\text{o.h.}}+3 e_{\text{int.}}=0\ .
\label{eq:Virial}
\end{equation}

Aleshores, aplicant l'Eq. \ref{eq:Virial} pels resultats de la Taula \ref{tab:res_a}, obtenim els valors de la Taula \ref{tab:res_d}.

\begin{table}[h!]
    \centering
    \begin{tabular}{|c|c|}
        \hline
        \rowcolor[HTML]{EFEFEF}
        \textbf{N} & $\mathbf{2e_{\text{cin.}}-2e_{\text{o.h.}}+3 e_{\text{int.}}}$ \\ \hline\hline
        \textbf{1000000} & $4\cdot10^{-4}$ \\ \hline
        \textbf{100000} & $5\cdot10^{-6}$ \\ \hline
        \textbf{10000} & $2\cdot10^{-5}$ \\ \hline
        \textbf{1000} & $4\cdot10^{-5}$ \\ \hline
        \textbf{100} & $4\cdot10^{-5}$ \\ \hline
    \end{tabular}
\caption{Resultats de fer la suma del teorema del Virial corresponent a la nostra equació de GP per diferents valors de $N$ amb les dades recollides en la Taula \ref{tab:res_a}. S'han aproximat al primer valor diferent de $0$ per saber els ordres de magnitud de diferència respecte al resultat del teorema del Virial.}
\label{tab:res_d}
\end{table}

Observem que els resultats difereixen del $0$ del teorema del Virial en $4$ fins a $6$ ordres de magnitud. Per tant, considerem que el teorema es compleix.

\end{document}